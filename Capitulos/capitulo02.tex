\chapter{Objetivos}

O objetivo principal deste trabalho é desenvolver um método de predição de estruturas secundárias utilizando autômatos celulares. Diferentemente dos métodos atuais, onde a predição é realizada em apenas dois estados, sequência -> estrutura secundária, o método proposto é iterativo, representando uma dinâmica de formação das estruturas secundárias. 

Neste trabalho, optamos por utilizar o modelo de autômatos celulares como o método iterativo. Entretanto, métodos recentes de aprendizado profundo poderiam ser utilizados como alternativa ou complemento do modelo proposto, desde que mantivessem implicitamente ou explicitamente o caracter iterativo. 

\section{Objetivos específicos}

O objetivo principal envolve os seguintes objetivos específicos:

\begin{enumerate}
	\item Preparação de um conjunto de treinamento composto da sequência de aminoácidos e das estruturas secundárias atribuídas por diferentes métodos a partir de uma grande variedade de estruturas proteicas;
	\item Implementação de autômatos celulares com estados que possibilitem representar a sequência de aminoácidos e os elementos de estrutura secundária. Assim, o estado inicial desses autômatos celulares representariam a sequência de aminoácidos e durante a evolução do autômato celular, os elementos de estrutura secundária surgiriam e se organizariam para formar a estrutura secundária proteica;
	\item Implementação de um método de otimização das regras de transição dos autômatos celulares como o objetivo de maximizar a acurácia do método de predição;
	\item Aplicação do método, análise e comparação dos resultados com outros métodos de predição.
\end{enumerate}

%%%%%% Pós correções do Paulão
% O objetivo principal deste trabalho será desenvolver um método de reconhecimento de
% enovelamento capaz de identificar, a partir da sequência de aminoácidos da proteína a qual se
% deseja modelar, estruturas proteicas resolvidas semelhantes a sua estrutura nativa. Tais
% estruturas podem ser tanto de proteínas que tenham evoluído divergentemente quanto
% convergentemente uma vez que não há uma comparação direta entre os resíduos das
% sequências, mas sim entre estruturas e padrões locais que emergem dessa estrutura primária. Este processo passa pela identificação correta da estrutura secundária da proteína a partir da sequência de aminoácidos
% Para atingir tal objetivo, delineamos especificamente os objetivos a seguir.

% Objetivos específicos

% 2. Aplicar este alfabeto a domínios proteicos obtidos do CATH e/ ou SCOP para
% criar uma representação das estruturas em dimensões reduzidas – 1D;

% 2.  Utilizar um conjunto de treinamento para ser usado na busca de regras de
% transição de estados para um autômato celular. Essas regras devem ser capazes de guiar a
% evolução do autômato celular de um estado inicial correspondente a sequência
% de aminoácidos da proteína até um estado que simbolize a estrutura secundária, mas representada em uma dimensão;

% 3. Testar as regras de transição selecionadas por apresentar melhor desempenho em
% proteínas não incluídas no conjunto de treinamento e assim, analisar a eficácia do
% autômato celular em obter informação estrutural a partir da sequência de
% aminoácidos das proteínas;


% 4. Utilizar o método de autômato celular com a melhor regra de transição para,
% quando aplicado em sequências de aminoácidos de proteínas sem estrutura
% conhecida, identificar os elementos de estrutura secundária bem como avaliar a dinâmica de sua formação.



%%%%%% Projeto
% O objetivo geral deste trabalho será desenvolver um método de reconhecimento de
% enovelamento capaz de identificar, a partir da sequência de aminoácidos da proteína a qual se
% deseja modelar, estruturas proteicas resolvidas semelhantes a sua estrutura nativa. Tais
% estruturas podem ser tanto de proteínas que tenham evoluído divergentemente quanto
% convergentemente uma vez que não há uma comparação direta entre os resíduos das
% sequências, mas sim entre estruturas e padrões locais que emergem dessa estrutura primária.
% Para atingir tal objetivo, delineamos especificamente os objetivos a seguir.

% Objetivos específicos
% 1. Criar um alfabeto com estados discretos e em menor número possível, para
% representar a conformações dos resíduos na estrutura proteica, buscando
% minimizar a perda de informação estrutural durante a redução de dimensões (3D
% %→ 1D). Exemplos de alfabetos 3D → 1D são os presentes: (1) no programa
% Enviroments (Verify3D) onde a redução é feita utilizando estados que combinam
% a estrutura secundária, a polaridade do ambiente e o grau de exposição ao
% solvente, totalizando 18 estados; (2) no programa FUGUE, onde há 4 classes para
% estruturas secundárias, 2 para acessibilidade ao solvente e 8 para ligações de
% hidrogênio, totalizando 64 estados; (3) no trabalho de Chellapa e Rose (2012) onde são utilizados 11 estados que representam regiões de ângulos diédricos da
% cadeia principal;

% 2. Aplicar este alfabeto a domínios proteicos obtidos do CATH e/ ou SCOP para
% criar uma representação das estruturas em dimensões reduzidas – 1D;

% 3. Utilizar a representação 1D de parte dos domínios obtidos do CATH/SCOP e
% assim criar um conjunto de treinamento para ser usado na busca de regras de
% transição para um autômato celular. Essas regras devem ser capazes de guiar a
% evolução do autômato celular de um estado inicial correspondente a sequência
% de aminoácidos da proteína até um estado que simbolize a estrutura
% tridimensional, mas representada unidimensionalmente, utilizando o alfabeto
% criado;

% 4. Testar as regras de transição selecionadas por apresentar melhor desempenho em
% proteínas não incluídas no conjunto de treinamento e assim, analisar a eficácia do
% autômato celular em obter informação estrutural a partir da sequência de
% aminoácidos das proteínas;

% 5. Utilizar o método de autômato celular com a melhor regra de transição para,
% quando aplicado em sequências de aminoácidos de proteínas sem estrutura
% conhecida, identificar domínios com enovelamento similar.