\chapter{Justificativa}

% PSLO não seesqueça de colocar o número de anos

Apesar de métodos de predição de estruturas secundárias serem estudados a quase 40 anos \cite{Chou1977a,Garnier1978} e, ao longo do tempo, terem atingido uma acurácia em torno de 80\% \cite{Faraggi2012a}, o método proposto por nós contém características não encontradas em outros métodos de predição de estrutura secundária.

% PSLO Zé: acho que vale a pena itemizar estas características para destacar pois são bem diferentes dos demais métodos 

Algumas dessas características são:

\begin{itemize}
	\item a utilização da sequência de aminoácidos completa ao invés do particionamento da sequência em janelas, como comumente utilizado em redes neurais;
	\item a característica iterativa do autômato celular, a qual proporciona a emergência de padrões complexos originados a partir de padrões simples;
	\item o uso somente da sequência de aminoácidos da proteína, ao invés de matrizes de substituição específica por posição (PSSM) ou descritores físico-químicos;
	\item a simplicidade na compreensão da informação contida nas regras de transição de um autômato celular capaz de predizer a formação de estruturas secundárias. 
\end{itemize}

Acreditamos que a possibilidade de obter informações que outros métodos de predição de estrutura secundária não fornecem tornam esse trabalho relevante cientificamente. Por exemplo, algumas possibilidades seriam a obtenção de informações sobre a dinâmica de formação das estruturas secundárias, ou até mesmo, sobre o enovelamento. Assim como aplicações práticas como a análise de perturbações causadas por mutações pontuais, a utilização no design de proteínas e a contribuição para métodos ab initio de predição de estruturas terciárias.

% , podendo fornecer informações que outros métodos de predição são incapazes. Por exemplo, acreditamos que não apenas da estrutura secundária na forma nativa,   

%  A aplicação de autômatos celulares no desenvolvimento de um novo método de
% reconhecimento de enovelamentos não visa ser apenas uma alternativa aos métodos atuais, mas
% também tem como objetivo a criação de um modelo que seja capaz de fornecer informações
% sobre a dinâmica do enovelamento de proteínas, ao permitir a análise de padrões que indicam
% pontos de início do enovelamento, de formação de estruturas locais e da propagação global
% dessas estruturas locais ao longo da sequência. O método poderá auxiliar também no estudo de
% como mutações podem afetar o enovelamento proteico, assim como no design de proteínas e
% contribuir também para a predição ab initio de estruturas proteicas.